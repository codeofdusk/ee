%packages
\usepackage[margin = 2cm]{geometry}
\usepackage{bbm}
\usepackage[shortlabels]{enumitem}
\usepackage{amsmath,amssymb,amsfonts,amsthm} 
\usepackage{url}
\usepackage{color} %Uncomment to enable color
\usepackage{framed} 
\usepackage{fullpage} 
\usepackage{graphicx}
\usepackage{listings} 
\usepackage{csvsimple} %For including raw data in .csv form (generated by programs, etc).
\usepackage{setspace} 
\doublespacing
\usepackage{booktabs} 
\usepackage{longtable}
\usepackage[british]{babel}
\usepackage[backend=biber,dateabbrev=false,style=ieee]{biblatex}
\addbibresource{EE.bib}
\usepackage{hyperref}
\usepackage{float}
\usepackage{pdfpages} %for embedding PDFs in your document
\usepackage [ toc,page ]{ appendix }
\usepackage{titlesec}
\titleformat{\chapter}[block]
  {\normalfont\huge\bfseries}{\thechapter.}{1em}{\Huge}
\titlespacing*{\chapter}{0pt}{-19pt}{0pt}
%new environments, these aren't IB specific (or made by me) but might be useful.
\newenvironment{solution}{\bigskip\begin{leftbar}\par\noindent\textbf{Solution.} }{\hfill $\blacktriangle$ \end{leftbar} } 
%\newenvironment{definitions}{\begin{centering}$\begin{array}{lr}}{\end{array}$\end{centering}}

%new commands, not IB specific or made by me but might be useful.
\newcommand\floor[1]{\left\lfloor #1 \right\rfloor} 
%\newcommand\defi[2]{#1:&#2\\}
\newcommand\defi[2]{\bf{#1}:\hspace{1cm}#2\newline}
\newcommand\ceiling[1]{\left\lcein #1 \right\rceil} 
\newcommand{\bEnum}{\begin{enumerate}[1]}
\newcommand{\eEnum}{\end{enumerate}}
\newcommand{\bAlpha}{\begin{enumerate}[a]}
\newcommand{\eAlpha}{\end{enumerate}}
\newcommand{\bCapAlpha}{\begin{enumerate}[A]}
\newcommand{\eCapAlpha}{\end{enumerate}}

%If we forget to fill the line before starting a new enum environment, then bad things happens and the letter is written over the number we just wrote, so define an item to fix that.
\newcommand{\fItem}{\item \hfill \newline}

%listing options for listings package.
%this template was taken from http://stackoverflow.com/questions/586572/make-code-in-latex-look-nice and then modified a bit more.

\lstset{ 
language=python,                % choose the language of the code
basicstyle=\footnotesize,       % the size of the fonts that are used for the code
numbers=left,                   % where to put the line-numbers
numberstyle=\footnotesize,      % the size of the fonts that are used for the line-numbers
stepnumber=1,                   % the step between two line-numbers. If it is 1 each line will be numbered
numbersep=5pt,                  % how far the line-numbers are from the code
backgroundcolor=\color{white},  % choose the background color. You must add \usepackage{color}
showspaces=false,               % show spaces adding particular underscores
showstringspaces=false,         % underline spaces within strings
showtabs=false,                 % show tabs within strings adding particular underscores
frame=single,           % adds a frame around the code
tabsize=2,          % sets default tabsize to 2 spaces
captionpos=b,           % sets the caption-position to bottom
breaklines=true,        % sets automatic line breaking
breakatwhitespace=false,    % sets if automatic breaks should only happen at whitespace
escapeinside={\%*}{*)},          % if you want to add a comment within your code
keepspaces=true                 % keeps spaces in text, useful for keeping indentation of code 
}